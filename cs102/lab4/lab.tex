\documentclass{ffslides}
\ffpage{25}{1.777}
\usepackage[norules,nolineno]{lgrind}
\usepackage{color}
\begin{document}
\blankpage
\btext{.375}{.5}{.25}{\textcolor{red}{CS 102 Lab 4: Release the Gifs}}
\normalpage{Specification}{%
\btext{.1}{.3}{.6}{
The API for giphy can be found for free on the mashape marketplace
}
This program will allow the user to type in a search word and then 
search the Giphy website to find a gif related to the word they searched.
Once it finds the gif it will display it.
}
\normalpage{Analysis}{%
\btext{.15}{.2}{.6}{%
Each url is a hyperlink which leads to a webpage where you can learn 
more about each part of the program
}
We will use the \url{market.mashape.com/explore}mashape market to get 
the API for the Giphy website from \url{market.mashape.com/giphy/giphy}.
 We will the CURL API, from \url{curl.haxx.se/libcurl} to get the gif 
file. The gif will be animated
 using FLTK and Cairo from 
\url{www.fltk.org/doc-1.3/group__group__cairo.html}
}
\normalpage{Design}{%
\btext{.15}{.5}{.6}{%
Each of the implementation slides, and therefore each .cpp file, has 
one function in it.  The function shares the name of the file aswell.
}
\qi{\hyperlink{labh}{labh}}
\qii{Contains all shared headers and variables}
\qi{\hyperlink{makeSearchWindow}{makeSearchWindow}}
\qii{Makes a window with a search box that people can use to find Gif}
\qi{\hyperlink{getGifInfo}{getGifInfo}}
\qii{Get data, in json format, from Giphy API from mashape 
specific to the users keywords}
\qi{\hyperlink{extractGifInfo}{extractGifInfo}}
\qii{Given a string in json format with all the info about
 the gif and return the information}
\qi{\hyperlink{cbDisplay}{cbDisplay}}
\qii{Uses given keyword, finds appropriate gif, and displays the gif}
\qi{\hyperlink{main}{main}}
\qii{Use console to test getting and displaying a gif then using a 
GUI to let user choose their own file}
}
\input labhDoc
\input makeSearchWindowDoc
\input getGifInfoDoc
\input extractGifInfoDoc
\input makeDisplayWindowDoc
\input cbDisplayDoc
\input mainDoc
\normalpage{Test}{%
\btext{.65}{.01}{.25}{%
The user cannot type in spaces, but can use "+" as an alternative
}
\qi{The user can search whatever they want}
\qii{If the user searches for "cats" they will get a cat related gif}
\qii{if the user searches for "gravity+falls" you will get another 
related gif}
\putfig {.01}{.05}{.45}{figure1.eps}
\putfig {.47}{.05}{.45}{figure2.eps}
}
\end{document}
