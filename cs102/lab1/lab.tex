\documentclass{ffslides}
\ffpage{25}{1.7777}
\usepackage{lgrind}
\begin{document}
\blankpage%
\dtext{%
\centering \Large \color{blue} cs102 lab 1
}
\btext{.4}{.2}{.2}{
Specification
}
\ctext{.2}{.4}{.6}{%
\[
    payment = \frac{intRate \times \frac{principal}{payPerYear}}
     {1 - (\frac{intRate}{payPerYear} + 1)^{- payPerYear \times numYears}}
\]
}
\ctext{.2}{.6}{.6}{%
For example, the payment for interest rate of .09, principle of 1000, with 12 payments per year, and 5 years for loan.
\[
    payment = \frac{.09 \times \frac{1000}{12}}
     {1 - (\frac{.09}{12} + 1)^{- 12 \times 5}}
\]
}
\blankpage%
\btext{.4}{.1}{.2}{
Analysis
}
\ctext{.1}{.2}{.6}{%
\input graph
}
\blankpage%
\btext{.1}{.1}{.8}{
Design:

\qi{inputs: interest rate}
\qi{outputs: monthly payment}
\qi{process (convert inputs to outputs)}
\qii{Assume principal is \$1,000}
\qii{Payments per year is 12}
\qii{The loan will be repaind in 5 years}
\qii{ask user to enter the interest rate}
\qii{then call function f (which calls function pmt which use the
formula to calculate the monthly payment)}

}
\blankpage%
\btext{.4}{.1}{.2}{
Implementation
}
\ctext{.2}{.2}{.5}{%
\lgrindfile{labcpp}
}
\blankpage%
\btext{.4}{.1}{.2}{
Test: If user enters interest rate of 9\% and principal is \$1,000,
payments per year is 12, and number of years is 5:
}
\putfig{.2}{.5}{.5}{fig1}



\blankpage%

\end{document}
