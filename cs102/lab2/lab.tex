\documentclass{ffslides}
\ffpage{25}{1.7777}
\usepackage{lgrind}
\begin{document}
\blankpage%
\dtext{%
\centering \Large \color{red} cs102 lab 2
}
\btext{.435}{.2}{.111}{%
Specification
}
\btext{.1}{.3}{.2}{%
The "Payment" refers to the amount of each montly payment
}
\ctext{.2}{.4}{.6}{%
\[
    payment = \frac{intRate \times \frac{principal}{payPerYear}}
     {1 - (\frac{intRate}{payPerYear} + 1)^{- payPerYear \times numYears}}
\]
}
\ctext{.2}{.6}{.6}{%
For example, the payment for interest rate of .09, principle of 1000, with 12 payments per year, and 5 years for loan.
\[
    payment = \frac{.09 \times \frac{1000}{12}}
     {1 - (\frac{.09}{12} + 1)^{- 12 \times 5}}
\]
}
\blankpage%
\btext{.45}{.1}{.08}{%
Analysis
}
\btext{.4}{.6}{.3}{%
The graph shows the value of the montly payment as we change interest
rate if the principal is \$1,000, number of years is 5, and the payment
per year is 12
}
\ctext{.1}{.2}{.6}{%
\input graph
}
\blankpage%
\btext{.455}{.1}{.07}{%
Design
}
\btext{.37}{.51}{.35}{%
GUI is short for Graphic User Interface
}
\ctext{.2}{.2}{.6}{%
\qi{inputs: Principle, Interest rate, number of Payments per Year, 
number of Years}
\qi{inputs are put in through a Graphic User Interface, with 4 text 
boxes, one for each variable}
\qi{outputs: monthly payment}
\qi{outputs are displayed in an unalterable text box}
\qi{process (convert inputs to outputs)}
\qii{ask user to enter the all the information, in the respective text 
box in the GUI}
\qii{Press Calculate}
\qii{then call function f (which calls function pmt which use the
formula to calculate the monthly payment)}
\qii{use "RND" function to round the value to cents, and then display 
in the appropriate text box}
}
\blankpage%
\btext{.4}{.1}{.2}{%
Implementation lab.h
}
\btext{.01}{.2}{.15}{%
List of all Variables and functions
}
\ctext{.2}{.2}{.6}{%
\lgrindfile{labheader}
}

\blankpage%
\btext{.4}{.1}{.22}{%
Implementation labgui.cpp
}
\btext{.01}{.3}{.15}{%
Declarations of all FLTK variables
}
\ctext{.2}{.2}{.6}{%
\lgrindfile{labgui}
}


\blankpage%
\btext{.4}{.1}{.2}{%
Implementation lab.cpp
}
\btext{.01}{.4}{.15}{%
Make Window is define on clabgui2
}
\btext{.65}{.2}{.3}{%
The function "PMT" uses 4 double variables. a=principle,r=interest rate, 
ppy=payments per year, n=number of years. Then it inputs each variable
into the monthly payment equation returning the payment value.
}
\ctext{.2}{.15}{.6}{%
\lgrindfile{labsource}
}

\blankpage%
\btext{.375}{.1}{.25}{%
Implementation clabgui1.cpp
}
\btext{.4}{.3}{.45}{%
Rounding works by: \newline
\qi{1) Multiplying by 100}
\qi{2) Rounding the number to the nearest whole integer}
\qi{3) Dividing by 100 to return to dollars and cents}
}
\btext{.2}{.7}{.6}{%
p's value is composition of the "rnd" function of the "f" function of the 
values of r, p, ppy, and n.
}
\ctext{.2}{.2}{.6}{%
\lgrindfile{clabgui1}
}

\blankpage%
\btext{.375}{.1}{.25}{%
Implementation clabgui2.cpp
}
\btext{.01}{.3}{.15}{%
The Make Window Function on its own in it's entirety
}
\btext{.75}{.3}{.2}{%
The name of a cairo text box is repective to the variable that the 
input is saved to.
}
\ctext{.2}{.2}{.6}{%
\lgrindfile{clabgui2}
}

\blankpage%
\btext{.47}{.1}{.04}{%
Test
}
\ctext{.2}{.2}{.6}{%
\qi{User can enter any data they desire}
\qi{If user enters principal of \$1,000, interest rate of 9\%,
payments per year is 12, and number of years is 5:}
\qi{If user enters principal of \$10000, interest rate of 3\%,
payments per year is 6, and number of years is 4:}
}
\btext{.01}{.4}{.15}{%
The figures show the functions ability to round up and down
}
\putfig{.6}{.4}{.3}{fig1}
\putfig{.2}{.4}{.3}{fig2}




\blankpage%

\end{document}
