\documentclass{ffslides}
\ffpage{25}{1.7777}
\newrgbcolor{ffgridcolor}{1 0 0}
\usepackage[nolineno]{lgrind}
\begin{document}
\blankpage%
\dtext{%
\centering \Large \color{green} cs102 lab 3
}
\btext{.435}{.2}{.111}{%
Specification:
}
\btext{.1}{.3}{.2}{%
The "Payment" refers to the amount of each montly payment}
\ctext{.2}{.4}{.6}{%
\[
    payment = \frac{intRate \times \frac{principal}{payPerYear}}
     {1 - (\frac{intRate}{payPerYear} + 1)^{- payPerYear \times numYears}}
\]
}
\ctext{.2}{.6}{.6}{%
For example, the payment for interest rate of .09, principle of 1000, 
with 12 payments per year, and 5 years for loan.
\[
    payment = \frac{.09 \times \frac{1000}{12}}
     {1 - (\frac{.09}{12} + 1)^{- 12 \times 5}}
\]
}
\blankpage%
\btext{.45}{.1}{.08}{%
Analysis:
}
\btext{.6}{.45}{.2}{%
The graph will be drawn using the 2D Cairo graphics.  The 
graph will be created when the "Calculate" button is pressed.
}
\btext{.25}{.6}{.3}{%
The graph shows the value of the montly payment as we change interest
rate if the principal is \$1,000, number of years is 5, and the payment
per year is 12
}
\ctext{.1}{.2}{.6}{%
\input graph
}
\blankpage%
\btext{.455}{.1}{.07}{%
Design:
}
\btext{.05}{.4}{.1}{%
Clicking the file name will send you straight to its designated 
implementation page
}
\ctext{.2}{.2}{.6}{%
The program is split into several modules:
\qi{\hyperlink{labh}{lab.h}}
\qii{contains all the include files and variables shared amongst
multiple cpp files}
\qi{\hyperlink{labgui}{labgui.cpp}}
\qii{declares all the FLTK Variables for the calculation GUI}
\qi{\hyperlink{clabgui1}{clabgui1.cpp}}
\qii{rnd: rounds off the value of the payment into cents}
\qii{cb\_calculate: programs the "Calculate" button}
\qi{\hyperlink{clabgui2}{clabgui2.cpp}}
\qii{the actual code just for the Calculation Window}
\qi{\hyperlink{labgraph}{labgraph.cpp}}
\qii{Makes the window for the graph to go in}
\qi{\hyperlink{cbDrawGraph}{cbDrawGraph.cpp}}
\qii{Creates the variables for the graph}
\qii{Calls the functions to:}
\qiii{Draw X-Axis}
\qiii{Draw Y-Axis}
\qiii{Plot the Point}
}

\blankpage%
\btext{.43}{.11}{.13}{%
Design (cont.):
}
\btext{.85}{.4}{.1}{%
Clicking the file name will send you straight to its designated 
implementation page
}
\ctext{.2}{.2}{.6}{%
\qi{\hyperlink{drawXAxis}{drawXAxis.cpp}}
\qii{Draws the X-Axis, with ticks and labels}
\qi{\hyperlink{drawYAxis}{drawYAxis.cpp}}
\qii{Draws the Y-Axis, with ticks and labels}
\qi{\hyperlink{plotPoint}{plotPoint.cpp}}
\qii{Places a circle at the values of (interest rate,payment value)}
\qi{\hyperlink{lab}{lab.cpp}}
\qii{main: Makes the main window and then contorl is move to FLTK}
\qii{f: Calls pmt with user values for principal, payments per year,
 interest rate, and number of years.}
\qii{pmt: Uses the formula to calculate and display the payment}
}

\blankpage%
\makesection[labh]{labh}
\btext{.4}{.1}{.2}{%
Implementation lab.h
}
\btext{.01}{.2}{.15}{%
List of all Variables and functions
}
\ctext{.2}{.13}{.6}{%
\lgrindfile{labheader}
}

\blankpage%
\makesection[labgui]{labgui}
\btext{.37}{.1}{.22}{%
Implementation labgui.cpp
}
\btext{.01}{.3}{.15}{%
Declarations of all FLTK variables
}
\ctext{.2}{.2}{.6}{%
\lgrindfile{labgui}
}
\blankpage%
\makesection[clabgui1]{clabgui1}
\btext{.37}{.1}{.25}{%
Implementation clabgui1.cpp
}
\btext{.4}{.32}{.45}{%
Rounding works by: \newline
\qi{1) Multiplying by 100}
\qi{2) Rounding the number to the nearest whole integer}
\qi{3) Dividing by 100 to return to dollars and cents}
}
\btext{.2}{.85}{.6}{%
p's value is composition of the "rnd" function of the "f" function of the 
values of r, p, ppy, and n.
}
\ctext{.2}{.2}{.6}{%
\lgrindfile{clabgui1}
}

\blankpage%
\makesection[clabgui2]{clabgui2}
\btext{.37}{.1}{.25}{%
Implementation clabgui2.cpp
}
\btext{.01}{.3}{.15}{%
The Make Window Function on its own in it's entirety
}
\btext{.75}{.3}{.2}{%
The name of a cairo text box is repective to the variable that the 
input is saved to.
}
\ctext{.2}{.2}{.6}{%
\lgrindfile{clabgui2}
}

\blankpage%
\makesection[labgraph]{labgraph}
\btext{.37}{.1}{.25}{%
Implementation labgraph.cpp
}
\btext{.05}{.3}{.07}{%
Makes a window called "cg" for the graph
}
\ctext{.2}{.2}{.6}{%
\lgrindfile{labgraph}
}

\blankpage%
\makesection[cbDrawGraph]{cbDrawGraph}
\btext{.35}{.1}{.3}{%
Implementation cbDrawGraph.cpp
}
\ctext{.2}{.15}{.6}{%
\lgrindfile{labdrawgraph}
}
\btext{.1}{.6}{.8}{%
\input xAxis
}
\btext{.7}{.5}{.2}{
Diagram shows length of each variable made in this file
}

\blankpage%
\makesection[drawXAxis]{drawXAxis}
\btext{.35}{.1}{.3}{%
Implementation drawXAxis.cpp
}
\btext{.4}{.4}{.2}{%
Creates a straight line to represent the X-Axis
}
\btext{.61}{.62}{.2}{%
"i" increments by 2 so the scale of the X-Axis will be 2 per tick
}
\ctext{.2}{.15}{.6}{%
\lgrindfile{drawXaxis}
}

\blankpage%
\makesection[drawYAxis]{drawYAxis}
\btext{.35}{.1}{.3}{%
Implementation drawYAxis.cpp
}
\btext{.4}{.4}{.2}{%
Creates a vertical line for the Y-Axis
}
\btext{.61}{.62}{.2}{%
Creates ticks in the same way as the X-Axis but now it increments by
5 so to increase the range of the graph
}
\ctext{.2}{.15}{.6}{%
\lgrindfile{drawYaxis}
}

\blankpage%
\makesection[plotPoint]{plotPoint}
\btext{.35}{.1}{.3}{%
Implementation plotPoint.cpp
}
\ctext{.2}{.15}{.6}{%
\lgrindfile{plotPoint}
}
\btext{.6}{.5}{.3}{%
Draws a circle at the point (r value,p value) from the "pmt" function
}

\blankpage%
\makesection[lab]{lab}
\btext{.4}{.1}{.2}{%
Implementation lab.cpp
}
\ctext{.55}{.65}{.6}{%
\[
    payment = \frac{r \times \frac{a}{ppy}}
     {1 - (\frac{r}{ppy} + 1)^{- ppy \times n}}
\]
}
\btext{.65}{.2}{.3}{%
The function "PMT" uses 4 double variables. a=principle,r=interest rate, 
ppy=payments per year, n=number of years. Then it inputs each variable
into the monthly payment equation returning the payment value.
}
\ctext{.2}{.15}{.6}{%
\lgrindfile{labsource}
}



\blankpage%
\btext{.47}{.1}{.04}{%
Test
}
\ctext{.2}{.2}{.6}{%
\qi{User can enter any data they desire}
\qi{If user enters principal of \$1,000, interest rate of 5\%,
payments per year is 12, and number of years is 5 and its point on
the graph is:}
\qi{If user enters principal of \$1,000, interest rate of 12\%,
payments per year is 6, and number of years is 10 and its point on
the graph is:}
}
\btext{.01}{.4}{.15}{%
The figures show the functions ability to graph any point between (0,0)
to (16,40)
}
\putfig{.6}{.4}{.37}{test1}
\putfig{.2}{.4}{.37}{test}




\blankpage%

\end{document}
